%%%%%%%%%%%%%%%%%%%%%%%%%%%%%%%%%%%%%%%%%%%%%%%%%%%%%%%%%%%%%%%%%%%%%%%%%%%%%%%%%%%%%%%%%%%%%%%%%%%%%%
% (C) 2015, by Robert Hemstedt <CC-BY-SA 3.0>                                                        %
% This work is licensed under a Creative Commons Attribution-ShareAlike 3.0 Unported License.        %
% To view a copy of this license, visit http://creativecommons.org/licenses/by-sa/3.0/.              %
%%%%%%%%%%%%%%%%%%%%%%%%%%%%%%%%%%%%%%%%%%%%%%%%%%%%%%%%%%%%%%%%%%%%%%%%%%%%%%%%%%%%%%%%%%%%%%%%%%%%%%
\documentclass[11pt,a4paper]{article}
\usepackage[utf8x]{inputenc}
%\usepackage{ucs}
\usepackage[left=2.0cm, right=2.0cm, top=2.0cm, bottom=2.0cm]{geometry}
\usepackage{amsmath}
\usepackage{amsfonts}
\usepackage[ngerman]{babel}
\usepackage{bbm}
\usepackage{amssymb}
\usepackage[amsthm,thmmarks]{ntheorem}
%\usepackage{tabularx}
\usepackage[arrow, matrix, curve]{xy}
\usepackage{graphicx}
\usepackage{wrapfig}
\usepackage{color}
\usepackage{url}
\usepackage{hyperref}
% b) Lemma, Satz, Theorem usw.
\makeatletter
\renewtheoremstyle{plain}%
  {\item[\hskip\labelsep \theorem@headerfont ##1\ ##2:]}%
  {\item[\hskip\labelsep \theorem@headerfont ##1~##2~##3:]\mbox{}}
\makeatother

\theoremstyle{plain}
\newtheorem{Theorem}{Theorem}[section]
\newtheorem*{Satz}[Theorem]{Satz}
\newtheorem{Prop}[Theorem]{Proposition}
\newtheorem{Lemma}[Theorem]{Lemma}
\newtheorem{Korollar}[Theorem]{Korollar}
\newtheorem{Definition}[Theorem]{Definition}
\newtheorem*{Folgerung}[Theorem]{Folgerung}
\newtheorem*{Behauptung}[Theorem]{Behauptung}
\newtheorem{bez}[Theorem]{Bezeichnung}

\theorembodyfont{\upshape}
\newtheorem{Bemerkung}[Theorem]{Bemerkung}
\newtheorem{Beispiel}[Theorem]{Beispiel}

\newcommand{\herv}[1]{{\emph{\textbf{#1}}}}
\newcommand{\N}{\mathbb{N}}
\newcommand{\R}{\mathbb{R}}
\newcommand{\Z}{\mathbb{Z}}
\newcommand{\Q}{\mathbb{Q}}
\newcommand{\C}{\mathbb{C}}
\newcommand{\Ch}{\hat{\C}}
\newcommand{\cupdot}{\mathbin{\dot{\cup}}}

\def\presuper#1#2%
  {\mathop{}%
   \mathopen{\vphantom{#2}}^{#1}%
   \kern-\scriptspace%
   #2}

\numberwithin{equation}{section}

\newsavebox{\fmbox}
\newenvironment{fmpage}[1]
{\begin{lrbox}{\fmbox}\begin{minipage}{#1}}
{\end{minipage}\end{lrbox}\fbox{\usebox{\fmbox}}}

\linespread{1.05}
\author{Robert Hemstedt \\ \texttt{r@twopi.eu}}
\date{27. April 2015}
\title{Begleitmaterial Seminarvortrag Fast Multipole Methode für Potentiale in 2D}

\begin{document}
\maketitle
\textit{Dieses Werk unterliegt der Creative Commons Attribution-ShareAlike 3.0 Unported License.}\\
Die Folien des Vortrags sowie dieses Begleitmaterial lassen sich in meinem \texttt{GitHub}-Repository \\ {\url{https://github.com/euklid/BachelorThesis/tree/master/FirstPresentation} finden.
\section{Bezeichnungen}
\begin{tabular}{ll}
$x_1,\ldots,x_m,  z_1,\ldots,z_m$  & gegebene geladene Punkte, \\
$y_1,\ldots,y_n$  & gegebene Punkte, an denen das Potential $\Phi(y_i)$ berechnet werden soll, \\
$q_1,\ldots, q_m$  & Ladungen der Punkte $x_1,\ldots,x_m$, \\
$q(z_i)$ & Ladung des Punktes $z_i$, \\
$z_0$ & Quellpunkt, \\
$z_c, z_{c'}$ & (Multipol-)Expansionspunkte, \\
$z_L, z_{L'}$ & Lokale Expansionspunkte,\\
$p$ & Anzahl Summanden in der Multipolexpansion
\end{tabular}
\section{Fast Multipole Method -- Formeln}
\begin{itemize}
\item Potential an $y_j$ durch Ladung $q_i$ an $x_i$: \[\Phi_{x_i}(y_j) = -q_i\log(||y_j-x_i||)\]
\item Kernel: \[
\text{reeller Kernel: } G(\mathbf{y},\mathbf{x}) := -\log(||\mathbf{x}-\mathbf{y}||), \qquad \text{komplexer Kernel: }G (z_0,z) = -\log(z_0-z) \]
\item (1) Kernelexpansion: \begin{equation*}
G(z_0,z) = \sum_{k=0}^\infty {O_k(z_0-z_c)I_k(z-z_c)},
\end{equation*}
Wobei 
\begin{equation*}
I_k(z)=\frac{z^k}{k!}, k\geq 0 \qquad \text{ sowie } \qquad O_k(z)=\frac{(k-1)!}{z^k}, k\geq 1; \text{ und } O_0(z)=-\log(z).
\end{equation*}
\item (2) Multipolexpansion:\[\sum_{i=1}^m {G(z_0,z_i)q(z_i)} = \sum_{k=0}^\infty O_k(z_0-z_c)M_k(z_c) \approx \sum_{k=0}^p O_k(z_0-z_c)M_k(z_c)\]
\item (3) Momente:\[M_k(z_c) =  \sum_{i=1}^m I_k(z_i-z_c) q(z_i)\]
\item (4) M2M-Translation:  \[ M_k(z_c') = \sum_{l=0}^k I_{k-l}(z_c-z_c')M_l(z_c) \]
\item (5) Lokale Expansion: \[ \sum_{i=1}^m {G(z_0,z_i)q(z_i)} = \sum_{l=0}^\infty L_l(z_L)I_l(z_0-z_L)\approx \sum_{l=0}^p L_l(z_L)I_l(z_0-z_L)\]
\item (6) M2L-Translation: \[ L_l(z_L)=\sum_{k=0}^\infty O_{l+k}(z_L-z_c)M_k(z_c)\approx(-1)^l\sum_{k=0}^p O_{l+k}(z_L-z_c)M_k(z_c)\]
\item (7) L2L-Translation:\[L_l(z_{L'})=\sum_{s=0}^{p-l} I_s(z_{L'}-z_L)L_{l+s}(z_L)\]
\end{itemize}
\begin{figure}
\centering
\def\svgwidth{0.5\textwidth}
\input{moments.pdf_tex}
\caption{Lagesituation für Multipolexpansion}
\end{figure}


\appendix
\section{Quellen}
\nocite{*}
\bibliographystyle{plain}
\bibliography{sources}

\end{document}
