\documentclass{beamer}
\usepackage[utf8x]{inputenc}
\usepackage{amsmath}
\usepackage{amsfonts}
\usepackage{amssymb}
%\usepackage{theorem}
\usepackage{graphicx}
\usepackage{url}
\usepackage[ngerman]{babel}
\usepackage{hyperref}
\usetheme{Madrid}
%\usecolortheme{dolphin}
\usefonttheme{structurebold}
\setbeamercovered{transparent}
\setbeamertemplate{footline}[frame number]
\setbeamertemplate{caption}[numbered]
%\setbeamertemplate{navigation symbols}{}%remove navigation symbols
\newcommand{\grad}{^{\circ}}
\title{Fast Multipole Method für Potentiale in 2D}
\author{Robert Hemstedt}
\institute[Universität Bonn -- Bonn]{
\inst{}{Rheinische Friedrich Wilhelms-Universität Bonn -- Bonn, \newline \newline \tiny{betreut durch Prof. Dr. Schweitzer und Prof. Dr. Klein}}}

\date[27.04.2014]{27. April 2014}

\begin{document}
\large
\begin{frame}
\titlepage
\vfill
\begin{center}
\includegraphics[scale=0.5]{by-sa.png}\\
\tiny{This work is licensed under a Creative Commons Attribution-ShareAlike 4.0 International License.}
\end{center}
\end{frame}

\begin{frame}
\frametitle{Gliederung}
\tableofcontents
\end{frame}

\section{Motivation}
\begin{frame}
\frametitle{Motivation}
\begin{Beispiel}

In der Ebene seien $m$ Punkte $\{x_1,\ldots,x_m\}$ gegeben, die jeweils die Ladungen  $\{q_1,\ldots, q_m\}$ tragen. Zusätzlich seien $n$ Punkte $\{y_1,\ldots,y_n\}$ in der Ebene gegeben, an denen die Potenziale $\Phi(y_i)$ $(i=1,\ldots,n)$ berechnet werden sollen. Der Einfachheit halber gelte $x_i\neq y_j$ für alle $i=1\ldots m, j=1,\ldots,n$.
\end{Beispiel}
\begin{alertblock}{Frage}
In welcher Zeit können alle Potentiale an den ausgewählten Punkten berechnet werden?
\end{alertblock}
\end{frame}

\section{Klassischer Ansatz}
\frame{
\frametitle{Klassischer Ansatz -- Direktes Ausrechnen}
\begin{Fakt}
Potential: $\Phi_{x_i}(y_j) = -q_i\log(||y_j-x_i||)$
\end{Fakt}
Also ergeben sich für alle Potentiale an den Punkten $y_j, j=1,\ldots,n$: 
\begin{equation}\nonumber
\sum_{i=1}^m {\Phi_{x_i}(y_j)} \qquad \text{für alle }i=1,\ldots,n
\end{equation}
Also insgesamt $O(mn)$ Summationen für alle Potentiale. Insbesondere $O(m^2)$, wenn $m=n$.
}

\section{Fast Multipole Method für Potentiale}
\frame{
\frametitle{Fast Multipole Method für Potentiale}
Die Funktion $\Phi_{x_i}(y_j)$ bezeichnet man in manchen Fällen auch als \emph{Kernel}. Bezeichne: 
\begin{equation}\nonumber
G(\mathbf{y},\mathbf{x}) := -\log(||\mathbf{x}-\mathbf{y}||)\text{ und }q(\mathbf{x}_i):=q_i\text{, also }\Phi_{\mathbf{x}_i}{\mathbf{y}_j}=K(\mathbf{y}_j,\mathbf{x}_i)q(\mathbf{x}_i)
\end{equation}
Ein Trick der FFM ist es, das Problem in die Komplexe Ebene zu legen. Identifiziere: 
\begin{align*}
\mathbf{x}=(\mathbf{x}_1,\mathbf{x}_2) \quad &\Rightarrow \quad z = \mathbf{x}_1 + i\mathbf{x}_2, \\
\mathbf{y}=(\mathbf{y}_1,\mathbf{y}_2) \quad &\Rightarrow \quad z_0=\mathbf{y}_1 + i \mathbf{y}_2, \\
G(\mathbf{y},\mathbf{x}) = \Re(G(z_0,z))\text{, mit }& G (z_0,z) = -\log(z_0-z),
\end{align*}
wobei wir hier den komplexen (Hauptzweig des) Logarithmus verwenden.
}

\begin{frame}{Expansion des $G$-Kernels}
In dieser Notation berechnet man: 
\begin{equation}\label{potential}
\sum_{i=1}^m {G(y_j,x_i)q(x_i)} \qquad \text{für alle }i=1,\ldots,n, 
\end{equation}
und nimmt den \emph{Realanteil}. Dabei sind die Punkte $(x_i)_{i=1,\ldots,m}$, $(y_j)_{j=1,\ldots,n}$ entsprechend in die Komplexe Ebene eingebettet.\\
Die Idee der Multipolexpansion ist es nun für einen Expansionspunkt $\mathbf{y}_c$ den Kernel zu schreiben als 
\begin{equation}
\nonumber
G(\mathbf{y},\mathbf{x}) = \sum_i G_i^y(\mathbf{y}_c,\mathbf{y})G_i^x(\mathbf{y}_c,\mathbf{x})
\end{equation}
\begin{alertblock}

$G_i^x(\mathbf{y_c},\mathbf{x})$ ist \emph{unabhängig} von $\mathbf{y}$, kann also wiederverwendet werden!

\end{alertblock}
\end{frame}

\begin{frame}{Expansion des $G$-Kernels}
\begin{alertblock}{Konvention}
Ab sofort arbeiten wir nur noch in $\mathbb{C}$.
\end{alertblock}
\begin{Lemma}
Es seien der \emph{Quellpunkt} $z_0$, der \emph{Feldpunkt} $z$ und ein \emph{Expansionspunkt} $z_c$ nahe $z$ gegeben, d.h. $|z-z_c| \ll |z_0-z_c|$. Dann gilt: 
\begin{equation}\label{multiexp}
G(z_0,z) = \sum_{k=0}^\infty {O_k(z_0-z_c)I_k(z-z_c)}.
\end{equation}
Wobei 
\begin{align*}
 \nonumber
I_k(z)&=\frac{z^k}{k!}, k\geq 0 \text{ sowie } \\ 
O_k(z)&=\frac{(k-1)!}{z^k}, k\geq 1; \text{ und } O_0(z)=-\log(z).
\end{align*}
\end{Lemma}
\end{frame}

\begin{frame}{Expansion des $G$-Kernels}
\begin{Beweis}
\[
G(z_0,z) = -\log(z_0-z) = -\left[\log(z_0-z_c) + \log \left(1-\frac{z-z_c}{z_0-z_c}\right)\right]
\]
Taylor-Expansion von \[\log(1-\xi) = -\sum_{k=1}^\infty {\frac{\xi^k}{k} }, |k| < 1\] liefert das gewünschte Ergebnis mit $\xi = \frac{z-z_c}{z_0-z_c}$.
\end{Beweis}
\begin{alertblock}{}
Es ist einer der Schlüsselpunkte der FMM, dass durch Gleichung \eqref{multiexp} im $G$-Kernel $z_0$ und $z$ durch $z_c$ separiert werden.
\end{alertblock}
\end{frame}

\begin{frame}{Multipolexpansion}
\small
\begin{Satz}{Multipolexpansion}
Es seien die Punkte $z_1,\ldots, z_m$ mit Ladungen $q(z_1),\ldots,q(z_m)$ gegeben. Weiterhin seien $z_0$ und $z_c$ derart, dass $|z_i-z_c| \ll |z_0-z_c|$ für alle $i=1,\ldots,m$. Dann gilt:
\begin{equation}\nonumber
\sum_{i=1}^m {G(z_0,z_i)q(z_i)} = \sum_{i=1}^m \left[\sum_{k=0}^\infty O_k(z_0-z_c)I_k(z_i-z_c)\right] q(z_i);
\end{equation}
also die \emph{Multipolexpansion}:
\begin{equation}\label{multipolexp}
\sum_{i=1}^m {G(z_0,z_i)q(z_i)} = \sum_{k=0}^\infty O_k(z_0-z_c)M_k(z_c),
\end{equation}
mit den \emph{Momenten}:
\begin{equation}\label{moments}
M_k(z_c) =  \sum_{i=1}^m I_k(z_i-z_c) q(z_i).
\end{equation}
\end{Satz}
\end{frame}

\begin{frame}{Momente}
\begin{block}{Der Überlegung wert}
Die Momente \[ M_k(z_c) =  \sum_{i=1}^m I_k(z_i-z_c) q(z_i) \] sind nur von den gegebenen Ladungen $x_1,\ldots,x_m$ und dem Expansionspunkt $z_c$ abhängig und brauchen nicht nochmal berechnet werden, wenn $z_0$ seine Lage ändert!
Später werden die $z_c$ Zentren von Zellen eines Gitters sein, sodass die Momente $M_k(z_c)$ häufig wiederverwendet werden können. Dies ist ein \emph{Schlüsselfaktor}, warum die FMM so schnell ist!
\end{block}

\end{frame}
\begin{frame}{Fehler der Multipolexpansion}
In \eqref{multipolexp} tritt eine \emph{unendliche} Summe auf. Fehler in der Fast Multipole Method können durch die Anzahl der ausgewerteten Terme in \eqref{multipolexp} kontrolliert werden.

\end{frame}

\subsection{FFM-Algorithmus}
\frame{
\frametitle{FFM-Algorithmus}
}

\section{Vorteile der FFM}
\frame{
\frametitle{Vorteile der FFM}
}


\end{document}